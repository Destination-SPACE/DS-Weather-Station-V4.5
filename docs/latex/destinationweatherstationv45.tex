%% Generated by Sphinx.
\def\sphinxdocclass{report}
\documentclass[letterpaper,10pt,english]{sphinxmanual}
\ifdefined\pdfpxdimen
   \let\sphinxpxdimen\pdfpxdimen\else\newdimen\sphinxpxdimen
\fi \sphinxpxdimen=.75bp\relax
\ifdefined\pdfimageresolution
    \pdfimageresolution= \numexpr \dimexpr1in\relax/\sphinxpxdimen\relax
\fi
%% let collapsible pdf bookmarks panel have high depth per default
\PassOptionsToPackage{bookmarksdepth=5}{hyperref}

\PassOptionsToPackage{booktabs}{sphinx}
\PassOptionsToPackage{colorrows}{sphinx}

\PassOptionsToPackage{warn}{textcomp}
\usepackage[utf8]{inputenc}
\ifdefined\DeclareUnicodeCharacter
% support both utf8 and utf8x syntaxes
  \ifdefined\DeclareUnicodeCharacterAsOptional
    \def\sphinxDUC#1{\DeclareUnicodeCharacter{"#1}}
  \else
    \let\sphinxDUC\DeclareUnicodeCharacter
  \fi
  \sphinxDUC{00A0}{\nobreakspace}
  \sphinxDUC{2500}{\sphinxunichar{2500}}
  \sphinxDUC{2502}{\sphinxunichar{2502}}
  \sphinxDUC{2514}{\sphinxunichar{2514}}
  \sphinxDUC{251C}{\sphinxunichar{251C}}
  \sphinxDUC{2572}{\textbackslash}
\fi
\usepackage{cmap}
\usepackage[T1]{fontenc}
\usepackage{amsmath,amssymb,amstext}
\usepackage{babel}



\usepackage{tgtermes}
\usepackage{tgheros}
\renewcommand{\ttdefault}{txtt}



\usepackage[Bjarne]{fncychap}
\usepackage{sphinx}

\fvset{fontsize=auto}
\usepackage{geometry}


% Include hyperref last.
\usepackage{hyperref}
% Fix anchor placement for figures with captions.
\usepackage{hypcap}% it must be loaded after hyperref.
% Set up styles of URL: it should be placed after hyperref.
\urlstyle{same}


\usepackage{sphinxmessages}
\setcounter{tocdepth}{1}



\title{Destination Weather Station v4.5}
\date{Mar 14, 2024}
\release{0.2.0}
\author{Destination SPACE Inc.\@{}}
\newcommand{\sphinxlogo}{\vbox{}}
\renewcommand{\releasename}{Release}
\makeindex
\begin{document}

\ifdefined\shorthandoff
  \ifnum\catcode`\=\string=\active\shorthandoff{=}\fi
  \ifnum\catcode`\"=\active\shorthandoff{"}\fi
\fi

\pagestyle{empty}
\sphinxmaketitle
\pagestyle{plain}
\sphinxtableofcontents
\pagestyle{normal}
\phantomsection\label{\detokenize{index::doc}}


\sphinxAtStartPar
An Open Source Remote Sensing Platform for National Taiwan Normal University

\sphinxhref{index.html}{\sphinxincludegraphics{{front}.png}}


\chapter{Getting Started}
\label{\detokenize{index:getting-started}}
\sphinxstepscope


\section{What is in the Kit?}
\label{\detokenize{hardware/the-kit:what-is-in-the-kit}}\label{\detokenize{hardware/the-kit:the-kit}}\label{\detokenize{hardware/the-kit::doc}}

\subsection{Kit Contents}
\label{\detokenize{hardware/the-kit:kit-contents}}\begin{itemize}
\item {} 
\sphinxAtStartPar
Destination Weather Station v4.5.2

\item {} 
\sphinxAtStartPar
Seeeduino XIAO RP2040 microcontroller w/ headers

\item {} 
\sphinxAtStartPar
128x64px OLED display w/ headers

\item {} 
\sphinxAtStartPar
SPDT power switch

\item {} 
\sphinxAtStartPar
AAA battery case

\item {} 
\sphinxAtStartPar
8GB micro\sphinxhyphen{}SD card

\item {} 
\sphinxAtStartPar
USB\sphinxhyphen{}A to USB\sphinxhyphen{}C cable

\end{itemize}

\sphinxstepscope


\section{Destination Weather Station v4.5 Assembly}
\label{\detokenize{hardware/assembly:destination-weather-station-v4-5-assembly}}\label{\detokenize{hardware/assembly:assembly}}\label{\detokenize{hardware/assembly::doc}}
\sphinxAtStartPar
The Destination Weather Station v4.5 is preassembled with the majority of the circuit components.

\sphinxAtStartPar
Before using the weather station, you will need to {\hyperref[\detokenize{hardware/assembly:id1}]{\sphinxcrossref{\DUrole{std,std-ref}{solder}}}} the {\hyperref[\detokenize{hardware/assembly:id2}]{\sphinxcrossref{\DUrole{std,std-ref}{XIAO RP2040 microcontroller}}}}, {\hyperref[\detokenize{hardware/assembly:id3}]{\sphinxcrossref{\DUrole{std,std-ref}{OLED display}}}}, and \DUrole{xref,std,std-ref}{power switch}. Follow the instructions below to solder your weather station. You can then {\hyperref[\detokenize{hardware/assembly:id4}]{\sphinxcrossref{\DUrole{std,std-ref}{connect the batteries}}}} and insert the microSD card.


\subsection{Soldering}
\label{\detokenize{hardware/assembly:soldering}}\phantomsection\label{\detokenize{hardware/assembly:id1}}
\sphinxAtStartPar
To solder the weather station, follow the instructions below. Tools required to complete these steps are:
\begin{itemize}
\item {} 
\sphinxAtStartPar
Soldering iron

\item {} 
\sphinxAtStartPar
Solder wire

\end{itemize}


\subsubsection{Power Switch}
\label{\detokenize{hardware/assembly:power-switch}}\phantomsection\label{\detokenize{hardware/assembly:switch}}\phantomsection\label{\detokenize{hardware/assembly:switch}}
\sphinxhref{assembly.html}{\sphinxincludegraphics{{switchOnly}.png}}

\sphinxAtStartPar
To solder the power switch to the weather station, insert it into the front side of the board. Flip the board over and solder the pins, starting with the two outer ones.


\subsubsection{XIAO RP2040}
\label{\detokenize{hardware/assembly:xiao-rp2040}}\phantomsection\label{\detokenize{hardware/assembly:id2}}
\sphinxAtStartPar
To solder the XIAO RP2040 to the weather station, begin by inserting the long side of the 1x7 header pins into the weather station board as shown below.

\sphinxhref{assembly.html}{\sphinxincludegraphics{{xiaoPinsOnly}.png}}

\sphinxAtStartPar
Next, flip the weather station over and solder all 14 pins to the weather station.

\sphinxAtStartPar
Finally, flip the weather station back over and solder the XIAO to the short side of the header pins. After you are done, it should look like the image below.

\sphinxhref{assembly.html}{\sphinxincludegraphics{{xiaoOnly}.png}}


\subsubsection{OLED Display}
\label{\detokenize{hardware/assembly:oled-display}}\phantomsection\label{\detokenize{hardware/assembly:id3}}
\sphinxAtStartPar
To solder the OLED display to the weather station, you will follow similar procedures to the steps above to solder the XIAO.

\sphinxAtStartPar
Begin by inserting the long side of the 1x4 header into the weather station board as shown below.

\sphinxhref{assembly.html}{\sphinxincludegraphics{{oledPinsOnly}.png}}

\sphinxAtStartPar
Next, flip the board over and solder the pins to the back.

\sphinxAtStartPar
Similarly, flip the board back over and solder the display to the front.

\sphinxhref{assembly.html}{\sphinxincludegraphics{{front}.png}}


\subsection{Final Assembly}
\label{\detokenize{hardware/assembly:final-assembly}}
\sphinxhref{assembly.html}{\sphinxincludegraphics{{iso}.png}}


\subsubsection{Batteries}
\label{\detokenize{hardware/assembly:batteries}}\phantomsection\label{\detokenize{hardware/assembly:id4}}
\sphinxAtStartPar
Insert three (3) AAA batteries into the battery pack. Connect the JST\sphinxhyphen{}PH connector to the matching connector on the weather station.


\chapter{Configuration}
\label{\detokenize{index:configuration}}
\sphinxstepscope


\section{Installing Arduino IDE}
\label{\detokenize{software/install-arduino-ide:installing-arduino-ide}}\label{\detokenize{software/install-arduino-ide:install-arduino-ide}}\label{\detokenize{software/install-arduino-ide::doc}}
\sphinxAtStartPar
To program the Destination Weather Station, you will need to download and install Arduino IDE.


\subsection{Download}
\label{\detokenize{software/install-arduino-ide:download}}
\sphinxAtStartPar
To begin, navigate to \sphinxurl{https://www.arduino.cc/en/software}.

\sphinxAtStartPar
Click \sphinxcode{\sphinxupquote{Windows Win 10 and newer, 64 bits}} under the most recent version of IDE. This may be different than the screenshot below

\sphinxhref{install-arduino-ide.html}{\sphinxincludegraphics{{arduino-ide-download-page}.png}}

\sphinxAtStartPar
When the page loads, click next

\sphinxhref{install-arduino-ide.html}{\sphinxincludegraphics{{arduino-ide-just-download-1}.png}}

\sphinxhref{install-arduino-ide.html}{\sphinxincludegraphics{{arduino-ide-just-download-2}.png}}


\subsection{Installation}
\label{\detokenize{software/install-arduino-ide:installation}}
\sphinxAtStartPar
To install Arduino IDE, navigate to where you saved your \sphinxcode{\sphinxupquote{arduino\sphinxhyphen{}ide\_2.x.x\_Windows\_64bit.exe}} file and follow the instructions in the install.


\section{Software Configuration}
\label{\detokenize{software/install-arduino-ide:software-configuration}}

\subsection{Board Manager}
\label{\detokenize{software/install-arduino-ide:board-manager}}
\sphinxAtStartPar
To properly configure IDE, the board manager for the XIAO RP2040 needs to be installed.

\sphinxAtStartPar
In Arduino IDE, navigate to \sphinxstylestrong{File \textgreater{} Preferences}. In the window, enter \sphinxcode{\sphinxupquote{https://github.com/earlephilhower/arduino\sphinxhyphen{}pico/releases/download/global/package\_rp2040\_index.json}} into the \sphinxstylestrong{Additional boards manager URLs:} text box.

\sphinxhref{install-arduino-ide.html}{\sphinxincludegraphics{{arduino-ide-preferences}.png}}

\sphinxAtStartPar
Click \sphinxstylestrong{OK}

\sphinxAtStartPar
Next, navigate to \sphinxstylestrong{Tools \textgreater{} Board: \textgreater{} Boards Manager…}. Search for \sphinxcode{\sphinxupquote{Raspberry Pi Pico/RP2040}} and click \sphinxstylestrong{INSTALL} on the option published by Earle F. Philhower, III.

\sphinxhref{install-arduino-ide.html}{\sphinxincludegraphics{{arduino-ide-boards-manager}.png}}


\subsection{Code Libraries}
\label{\detokenize{software/install-arduino-ide:code-libraries}}
\sphinxAtStartPar
Several libraries are needed  to upload code to the weather station.

\sphinxAtStartPar
To install these libraries, navigate to \sphinxstylestrong{Tools \textgreater{} Manage Libraries…}

\sphinxAtStartPar
In the search box enter the following library names and install each including any required dependancies.
\begin{itemize}
\item {} 
\sphinxAtStartPar
\sphinxcode{\sphinxupquote{Adafruit BME280 Library}}

\item {} 
\sphinxAtStartPar
\sphinxcode{\sphinxupquote{Adafruit LTR390 Library}}

\item {} 
\sphinxAtStartPar
\sphinxcode{\sphinxupquote{Adafruit SSD1306}}

\item {} 
\sphinxAtStartPar
\sphinxcode{\sphinxupquote{Adafruit VEML7700 Library}}

\item {} 
\sphinxAtStartPar
\sphinxcode{\sphinxupquote{ENS160 \sphinxhyphen{} Adafruit Fork}}

\item {} 
\sphinxAtStartPar
\sphinxcode{\sphinxupquote{Sensirion I2C SCD4x}}

\end{itemize}


\chapter{Sensors}
\label{\detokenize{index:sensors}}
\sphinxstepscope


\section{Destination Weather Station \sphinxhyphen{} Sensors}
\label{\detokenize{sensors/index:destination-weather-station-sensors}}\label{\detokenize{sensors/index:sensors}}\label{\detokenize{sensors/index::doc}}
\sphinxAtStartPar
The Destination Weather Station is equipped with a {\hyperref[\detokenize{sensors/bme280:bme280}]{\sphinxcrossref{\DUrole{std,std-ref}{Bosch BME280 combined humidity and pressure sensor}}}}.

\sphinxAtStartPar
In the expansion sensor kit, several more environmental sensors are provided to measure a wide variety of parameters. These include:
\begin{itemize}
\item {} 
\sphinxAtStartPar
{\hyperref[\detokenize{sensors/ens160:ens160}]{\sphinxcrossref{\DUrole{std,std-ref}{ScioSense ENS160 \sphinxhyphen{} Digital metal\sphinxhyphen{}oxide multi\sphinxhyphen{}gas sensor}}}}

\item {} 
\sphinxAtStartPar
{\hyperref[\detokenize{sensors/ltr390:ltr390}]{\sphinxcrossref{\DUrole{std,std-ref}{Lite\sphinxhyphen{}On LTR390 \sphinxhyphen{} Ambient light and ultravoiolet light sensor}}}}

\item {} 
\sphinxAtStartPar
{\hyperref[\detokenize{sensors/scd40:scd40}]{\sphinxcrossref{\DUrole{std,std-ref}{Sensirion SCD40 \sphinxhyphen{} Miniature CO2 sensor}}}}

\item {} 
\sphinxAtStartPar
{\hyperref[\detokenize{sensors/veml7700:veml7700}]{\sphinxcrossref{\DUrole{std,std-ref}{Vishay VEML7700 \sphinxhyphen{} High accuracy ambient light sensor}}}}

\end{itemize}

\sphinxstepscope


\section{BME280}
\label{\detokenize{sensors/bme280:bme280}}\label{\detokenize{sensors/bme280:id1}}\label{\detokenize{sensors/bme280::doc}}
\sphinxhref{bme280.html}{\sphinxincludegraphics{{MFG_BME280}.jpg}}

\sphinxAtStartPar
The BME280 combined humidity and pressure sensor is a classic device used for accuratly measureing temperature, pressure, and humidity.


\subsection{Humidity}
\label{\detokenize{sensors/bme280:humidity}}

\begin{savenotes}\sphinxattablestart
\sphinxthistablewithglobalstyle
\centering
\begin{tabulary}{\linewidth}[t]{TT}
\sphinxtoprule
\sphinxstyletheadfamily 
\sphinxAtStartPar
Parameter
&\sphinxstyletheadfamily 
\sphinxAtStartPar
Specifications
\\
\sphinxmidrule
\sphinxtableatstartofbodyhook
\sphinxAtStartPar
Operating Range
&
\sphinxAtStartPar
0 \sphinxhyphen{} 100 \%RH
\\
\sphinxhline
\sphinxAtStartPar
Absolute Accuracy
&
\sphinxAtStartPar
\(\pm\)3 \%RH
\\
\sphinxhline
\sphinxAtStartPar
Hysteresis
&
\sphinxAtStartPar
\(\pm\) \%RH
\\
\sphinxbottomrule
\end{tabulary}
\sphinxtableafterendhook\par
\sphinxattableend\end{savenotes}

\sphinxhref{bme280.html}{\sphinxincludegraphics{{rhGraph}.png}}


\subsection{Pressure}
\label{\detokenize{sensors/bme280:pressure}}

\begin{savenotes}\sphinxattablestart
\sphinxthistablewithglobalstyle
\centering
\begin{tabulary}{\linewidth}[t]{TT}
\sphinxtoprule
\sphinxstyletheadfamily 
\sphinxAtStartPar
Parameter
&\sphinxstyletheadfamily 
\sphinxAtStartPar
Specifications
\\
\sphinxmidrule
\sphinxtableatstartofbodyhook
\sphinxAtStartPar
Operating Range
&
\sphinxAtStartPar
0 \sphinxhyphen{} 1100 hPa
\\
\sphinxhline
\sphinxAtStartPar
Absolute Accuracy
&
\sphinxAtStartPar
\(\pm\) 1.0 hPa
\\
\sphinxhline
\sphinxAtStartPar
Relative Accuracy (700 \sphinxhyphen{} 900 hPa \& 25 \sphinxhyphen{} 40 °C)
&
\sphinxAtStartPar
\(\pm\) 0.12 hPa
\\
\sphinxbottomrule
\end{tabulary}
\sphinxtableafterendhook\par
\sphinxattableend\end{savenotes}


\subsection{Temperature}
\label{\detokenize{sensors/bme280:temperature}}

\begin{savenotes}\sphinxattablestart
\sphinxthistablewithglobalstyle
\centering
\begin{tabulary}{\linewidth}[t]{TT}
\sphinxtoprule
\sphinxstyletheadfamily 
\sphinxAtStartPar
Parameter
&\sphinxstyletheadfamily 
\sphinxAtStartPar
Specifications
\\
\sphinxmidrule
\sphinxtableatstartofbodyhook
\sphinxAtStartPar
Operating Range
&
\sphinxAtStartPar
0 \sphinxhyphen{} 65 °C
\\
\sphinxhline
\sphinxAtStartPar
Absolute Accuracy
&
\sphinxAtStartPar
0.5 °C
\\
\sphinxbottomrule
\end{tabulary}
\sphinxtableafterendhook\par
\sphinxattableend\end{savenotes}


\subsection{Functional Block Diagram}
\label{\detokenize{sensors/bme280:functional-block-diagram}}
\sphinxAtStartPar
Below is a functional block diagram for how the BME280 sensor works. The data for each parameter (humidity, pressure, and temperature) are measured using an analog sensor. The data is then converted from an analog voltage signal to a digital data signal. This data is then passed to the logic side of the sensor which sends and recieves data to/from the host device.

\sphinxhref{bme280.html}{\sphinxincludegraphics{{bme280BlockDiagram}.png}}

\sphinxstepscope


\section{ENS160}
\label{\detokenize{sensors/ens160:ens160}}\label{\detokenize{sensors/ens160:id1}}\label{\detokenize{sensors/ens160::doc}}
\sphinxhref{ens160.html}{\sphinxincludegraphics{{MFG_ENS160}.jpg}}

\sphinxAtStartPar
The ENS160 digital metal\sphinxhyphen{}oxide multi\sphinxhyphen{}gas sensor measures volitile organic compounds (VOCs), which can be used to measure total VOC content (TVOC), air quality index (AQI), and calculate an approximate value for carbon\sphinxhyphen{}dioxide concentration (eCO2).


\subsection{Specifications}
\label{\detokenize{sensors/ens160:specifications}}

\begin{savenotes}\sphinxattablestart
\sphinxthistablewithglobalstyle
\centering
\begin{tabular}[t]{*{4}{\X{1}{4}}}
\sphinxtoprule
\sphinxstyletheadfamily 
\sphinxAtStartPar
Parameter
&\sphinxstyletheadfamily 
\sphinxAtStartPar
Range
&\sphinxstyletheadfamily 
\sphinxAtStartPar
Resolution
&\sphinxstyletheadfamily 
\sphinxAtStartPar
Units
\\
\sphinxmidrule
\sphinxtableatstartofbodyhook
\sphinxAtStartPar
TVOC
&
\sphinxAtStartPar
0 \sphinxhyphen{} 65,000
&
\sphinxAtStartPar
1
&
\sphinxAtStartPar
ppb
\\
\sphinxhline
\sphinxAtStartPar
eCO2
&
\sphinxAtStartPar
400 \sphinxhyphen{} 65,000
&
\sphinxAtStartPar
1
&
\sphinxAtStartPar
ppm CO2 equivalent
\\
\sphinxhline
\sphinxAtStartPar
AQI\sphinxhyphen{}UBA
&
\sphinxAtStartPar
1 to 5
&
\sphinxAtStartPar
1
&\begin{itemize}
\item {} 
\end{itemize}
\\
\sphinxhline
\sphinxAtStartPar
AQI\sphinxhyphen{}EPA
&
\sphinxAtStartPar
0 \sphinxhyphen{} 500
&
\sphinxAtStartPar
1
&\begin{itemize}
\item {} 
\end{itemize}
\\
\sphinxbottomrule
\end{tabular}
\sphinxtableafterendhook\par
\sphinxattableend\end{savenotes}


\subsection{TVOC}
\label{\detokenize{sensors/ens160:tvoc}}
\sphinxAtStartPar
Total Volitile Organic Compounds (TVOC) is a useful parameter used to determine the quality of the air in a given area. This is incredibly important for urban environments and in indoor settings. This is typically measured in parts\sphinxhyphen{}per\sphinxhyphen{}billion (ppb), but in polluted environments, can reach into the parts\sphinxhyphen{}per\sphinxhyphen{}million (ppm) range.


\subsection{eCO2}
\label{\detokenize{sensors/ens160:eco2}}
\sphinxAtStartPar
The ENS160 sensor uses an algorythm to calculate an approximate value for carbon\sphinxhyphen{}dioxide (CO2) concentration. While the sensor cannot directly measure CO2, it’s algorythm can closely approximate this value, as seen in the graph below.

\sphinxhref{ens160.html}{\sphinxincludegraphics{{ens160_eco2}.jpg}}

\sphinxAtStartPar
In the graph, the calculated eCO2 measurment is being compared to a nondispersive infrared CO2 sensor. It can closely approximate the CO2 concentration, but it is not perfect.

\sphinxAtStartPar
In the graphs below, ScioSense has provided eCO2 measurments for different environments, such as in a bedroom, kitchen, and bathroom. These graphs show that different conditions, such as closing a window, cooking, or being in a room can increase CO2 concentrations.

\sphinxhref{ens160.html}{\sphinxincludegraphics{{ens160_eco2_bedroom}.jpg}}

\sphinxhref{ens160.html}{\sphinxincludegraphics{{ens160_eco2_kitchen}.jpg}}

\sphinxhref{ens160.html}{\sphinxincludegraphics{{ens160_eco2_bathroom}.jpg}}


\subsection{AQI}
\label{\detokenize{sensors/ens160:aqi}}
\sphinxAtStartPar
Air Quality Index (AQI) is the typical parameter used to judge the quality of air and how polluted it is. There are several different scales used to measure this value, but they are all based on the concentration of VOCs. For example, the United States uses the \sphinxhref{https://www.airnow.gov/aqi/aqi-basics/}{NowCast} algorythm developed by the Environmental Protection Agency (EPA) ranging from 0 to 500, the European Union’s \sphinxhref{https://airindex.eea.europa.eu/AQI/index.html}{European Environment Agency} (EEA) uses a scale ranging from 0 to 1250, and Taiwan’s \sphinxhref{https://airtw.epa.gov.tw/ENG/Information/Standard/AirQualityIndicator.aspx}{Ministry of Environment} uses a similar scale to the United States, ranging from 0 to 500. The figures below show each of these indexes.

\sphinxAtStartPar
Environmental Protection Agency (EPA)

\sphinxhref{ens160.html}{\sphinxincludegraphics{{aqi_us}.png}}

\sphinxAtStartPar
European Environment Agency (EEA)

\sphinxhref{ens160.html}{\sphinxincludegraphics{{aqi_eu}.png}}

\sphinxAtStartPar
Ministry of Environment

\sphinxhref{ens160.html}{\sphinxincludegraphics{{aqi_tw}.png}}

\sphinxstepscope


\section{LTR390}
\label{\detokenize{sensors/ltr390:ltr390}}\label{\detokenize{sensors/ltr390:id1}}\label{\detokenize{sensors/ltr390::doc}}
\sphinxhref{ltr390.html}{\sphinxincludegraphics{{MFG_LTR390}.jpg}}

\sphinxAtStartPar
The LTR390 optical sensor is a combined digital ambient light (ALS) and UVA sensor.


\subsection{Ambient Light Specifications}
\label{\detokenize{sensors/ltr390:ambient-light-specifications}}

\begin{savenotes}\sphinxattablestart
\sphinxthistablewithglobalstyle
\centering
\begin{tabular}[t]{*{6}{\X{1}{6}}}
\sphinxtoprule
\sphinxstyletheadfamily 
\sphinxAtStartPar
Parameter
&\sphinxstyletheadfamily 
\sphinxAtStartPar
Minimum
&\sphinxstyletheadfamily 
\sphinxAtStartPar
Typical
&\sphinxstyletheadfamily 
\sphinxAtStartPar
Maximum
&\sphinxstyletheadfamily 
\sphinxAtStartPar
Units
&\sphinxstyletheadfamily 
\sphinxAtStartPar
Condition
\\
\sphinxmidrule
\sphinxtableatstartofbodyhook
\sphinxAtStartPar
ALS Output Resolution
&
\sphinxAtStartPar
13
&
\sphinxAtStartPar
18
&
\sphinxAtStartPar
20
&
\sphinxAtStartPar
Bit
&
\sphinxAtStartPar
Programable for 13, 16, 17, 18, 19, 20 bit
\\
\sphinxhline
\sphinxAtStartPar
Dark Level Count
&\begin{itemize}
\item {} 
\end{itemize}
&
\sphinxAtStartPar
0
&
\sphinxAtStartPar
5
&
\sphinxAtStartPar
count
&
\sphinxAtStartPar
0 Lux, T\_ope=25°C, 18\sphinxhyphen{}bit range
\\
\sphinxhline
\sphinxAtStartPar
Calibrated Lux Error In Gain Range 3
&
\sphinxAtStartPar
\sphinxhyphen{}10
&\begin{itemize}
\item {} 
\end{itemize}
&
\sphinxAtStartPar
10
&
\sphinxAtStartPar
\%
&
\sphinxAtStartPar
While LED, 5000K, T\_ope=25°C
\\
\sphinxhline
\sphinxAtStartPar
ALS Accuracy
&
\sphinxAtStartPar
\sphinxhyphen{}25
&\begin{itemize}
\item {} 
\end{itemize}
&
\sphinxAtStartPar
25
&
\sphinxAtStartPar
\%
&
\sphinxAtStartPar
Across different light sources
\\
\sphinxbottomrule
\end{tabular}
\sphinxtableafterendhook\par
\sphinxattableend\end{savenotes}


\subsection{UVS Specifications}
\label{\detokenize{sensors/ltr390:uvs-specifications}}

\begin{savenotes}\sphinxattablestart
\sphinxthistablewithglobalstyle
\centering
\begin{tabular}[t]{*{6}{\X{1}{6}}}
\sphinxtoprule
\sphinxstyletheadfamily 
\sphinxAtStartPar
Parameter
&\sphinxstyletheadfamily 
\sphinxAtStartPar
Minimum
&\sphinxstyletheadfamily 
\sphinxAtStartPar
Typical
&\sphinxstyletheadfamily 
\sphinxAtStartPar
Maximum
&\sphinxstyletheadfamily 
\sphinxAtStartPar
Unit
&\sphinxstyletheadfamily 
\sphinxAtStartPar
Condition
\\
\sphinxmidrule
\sphinxtableatstartofbodyhook
\sphinxAtStartPar
UVS Output Resolution
&
\sphinxAtStartPar
13
&
\sphinxAtStartPar
18
&
\sphinxAtStartPar
20
&
\sphinxAtStartPar
Bit
&
\sphinxAtStartPar
Programmable for 13, 16,17, 18, 19, 20 bit
\\
\sphinxhline
\sphinxAtStartPar
UV Count
&\begin{itemize}
\item {} 
\end{itemize}
&
\sphinxAtStartPar
160
&\begin{itemize}
\item {} 
\end{itemize}
&
\sphinxAtStartPar
count
&
\sphinxAtStartPar
UV LED 310nm, T\_ope=25°C, 18\sphinxhyphen{}bit, Gain range = 18, Irradiance = 70uW/cm2
\\
\sphinxhline
\sphinxAtStartPar
UV Sensitivity
&\begin{itemize}
\item {} 
\end{itemize}
&
\sphinxAtStartPar
2300
&\begin{itemize}
\item {} 
\end{itemize}
&
\sphinxAtStartPar
Counts/UVI
&
\sphinxAtStartPar
Gain range = 18, 20\sphinxhyphen{}bit
\\
\sphinxhline
\sphinxAtStartPar
UVI Accuracy (UVI\textgreater{}5)
&
\sphinxAtStartPar
\sphinxhyphen{}20
&\begin{itemize}
\item {} 
\end{itemize}
&
\sphinxAtStartPar
20
&
\sphinxAtStartPar
\%
&
\sphinxAtStartPar
Gain Range = 18, 20bit
\\
\sphinxhline
\sphinxAtStartPar
UVI Accuracy (UVI\textless{}5)
&
\sphinxAtStartPar
\sphinxhyphen{}1
&\begin{itemize}
\item {} 
\end{itemize}
&
\sphinxAtStartPar
1
&
\sphinxAtStartPar
\%
&
\sphinxAtStartPar
Gain Range = 18, 20bit
\\
\sphinxbottomrule
\end{tabular}
\sphinxtableafterendhook\par
\sphinxattableend\end{savenotes}


\subsection{ALS Sensor}
\label{\detokenize{sensors/ltr390:als-sensor}}
\sphinxAtStartPar
The ambient light sensor is sensitive to a range of 450nm \sphinxhyphen{} 650nm and is centered at 535nm as seen in the graph below.

\sphinxhref{ltr390.html}{\sphinxincludegraphics{{ltr390_als_response}.png}}

\sphinxAtStartPar
Additionally, below is a graph showing the angle of incidence of the sensor, where a normalized count of 1 indicates full sensitivity.

\sphinxhref{ltr390.html}{\sphinxincludegraphics{{ltr390_als_aoi}.png}}


\subsection{UVS Sensor}
\label{\detokenize{sensors/ltr390:uvs-sensor}}
\sphinxAtStartPar
The UVS sensor peaks in sensitive for UVA wavelengths (315\sphinxhyphen{}400) with some moderate sensitivity to low\sphinxhyphen{}energy UVB wavelengths. The graph below indicates the the response of the UVS sensor, which has a range of 275nm \sphinxhyphen{} 400nm and is centered at 320nm.

\sphinxhref{ltr390.html}{\sphinxincludegraphics{{ltr390_uvs_response}.png}}


\subsection{The Ultra\sphinxhyphen{}Violet Spectrum}
\label{\detokenize{sensors/ltr390:the-ultra-violet-spectrum}}
\sphinxAtStartPar
The ultra\sphinxhyphen{}violet spectrum ranges from 10nm to 400nm. This is further subdivided into different bands of intensity, which are used in remote sensing and to indicate their danger.

\sphinxhref{ltr390.html}{\sphinxincludegraphics{{ultravioletSpectrum}.png}}


\subsection{UVA}
\label{\detokenize{sensors/ltr390:uva}}
\sphinxAtStartPar
UVA is the longest wavelength of the UV spectrum, ranging from 400nm to 315nm. UVA is the primary cause of sunburns encountered from being in the sun too long. This is because the longer wavelength allows it to penetrate deeper into the skin.


\subsection{UVB}
\label{\detokenize{sensors/ltr390:uvb}}
\sphinxAtStartPar
UVB is a the middle band of the UV spectrum and ranges from 315nm to 280nm. UVB is mostly absorbed by the ozone layer, but can still reach the earth’s surface. This is primarily seen at higher latitudes and elevations. Because UVB has a shorter wavelength, this is the primary cause of skin cancer and blistering from sunburns.


\subsection{UVC}
\label{\detokenize{sensors/ltr390:uvc}}
\sphinxAtStartPar
UVB is the highest energy band in the UV spectrum, ranging from 280nm to 100nm. UVC is entrily filtered out by the Earth’s atmosphere, so is no danger to life on the surface, but can become a risk durring air travel or in space. UVC is a type of ionizing radiation, meaning it easily kills cells. This is useful for UV disinfection lights.

\sphinxstepscope


\section{SCD40}
\label{\detokenize{sensors/scd40:scd40}}\label{\detokenize{sensors/scd40:id1}}\label{\detokenize{sensors/scd40::doc}}
\sphinxhref{scd40.html}{\sphinxincludegraphics{{MFG_SCD40}.jpg}}

\sphinxAtStartPar
THe SCD40 CO2 sensor uses photoacoustic nondispersive infrared (NDIR) technology to measure actual CO2 concentration, unlike the eCO2 measurement on the {\hyperref[\detokenize{sensors/ens160:ens160}]{\sphinxcrossref{\DUrole{std,std-ref}{ENS160}}}}. This works by measuring the attenuation of an infrared light shining through the gas onto an accoustic transducer. Along with the sample chamber there is a reference nitrogen sample used to compare the CO2 measurement to the known gas. This process works according to Beer\sphinxhyphen{}Lambert law which is this attenuation of the gas.

\sphinxhref{scd40.html}{\sphinxincludegraphics{{scd40_blockDiagram}.jpg}}


\subsection{CO2 Sensing Preformance}
\label{\detokenize{sensors/scd40:co2-sensing-preformance}}

\begin{savenotes}\sphinxattablestart
\sphinxthistablewithglobalstyle
\centering
\begin{tabular}[t]{*{3}{\X{1}{3}}}
\sphinxtoprule
\sphinxstyletheadfamily 
\sphinxAtStartPar
Parameter
&\sphinxstyletheadfamily 
\sphinxAtStartPar
Conditions
&\sphinxstyletheadfamily 
\sphinxAtStartPar
Value
\\
\sphinxmidrule
\sphinxtableatstartofbodyhook
\sphinxAtStartPar
Output range
&\begin{itemize}
\item {} 
\end{itemize}
&
\sphinxAtStartPar
0 \sphinxhyphen{} 40,000ppm
\\
\sphinxhline
\sphinxAtStartPar
Accuracy
&
\sphinxAtStartPar
400ppm \sphinxhyphen{} 2,000ppm
&
\sphinxAtStartPar
\(\pm\)(50ppm + 5\% of reading)
\\
\sphinxhline
\sphinxAtStartPar
Repeatability
&
\sphinxAtStartPar
Typical
&
\sphinxAtStartPar
\(\pm\)10ppm
\\
\sphinxbottomrule
\end{tabular}
\sphinxtableafterendhook\par
\sphinxattableend\end{savenotes}


\subsection{Humidty Sensing Preformance}
\label{\detokenize{sensors/scd40:humidty-sensing-preformance}}

\begin{savenotes}\sphinxattablestart
\sphinxthistablewithglobalstyle
\centering
\begin{tabular}[t]{*{3}{\X{1}{3}}}
\sphinxtoprule
\sphinxstyletheadfamily 
\sphinxAtStartPar
Parameter
&\sphinxstyletheadfamily 
\sphinxAtStartPar
Conditions
&\sphinxstyletheadfamily 
\sphinxAtStartPar
Value
\\
\sphinxmidrule
\sphinxtableatstartofbodyhook
\sphinxAtStartPar
Range
&\begin{itemize}
\item {} 
\end{itemize}
&
\sphinxAtStartPar
0 \%RH \sphinxhyphen{} 100 \%RH
\\
\sphinxhline
\sphinxAtStartPar
Accuracy
&
\sphinxAtStartPar
15°C \sphinxhyphen{} 35°C, 20 \%RH \sphinxhyphen{} 65 \%RH
&
\sphinxAtStartPar
\(\pm\)6 \%RH
\\
\sphinxhline
\sphinxAtStartPar
Accuracy
&
\sphinxAtStartPar
\sphinxhyphen{}10°C \sphinxhyphen{} 60°C, 0 \%RH \sphinxhyphen{} 100 \%RH
&
\sphinxAtStartPar
\(\pm\)9 \%RH
\\
\sphinxhline
\sphinxAtStartPar
Repeatability
&
\sphinxAtStartPar
Typical
&
\sphinxAtStartPar
\(\pm\)0.4 \%RH
\\
\sphinxbottomrule
\end{tabular}
\sphinxtableafterendhook\par
\sphinxattableend\end{savenotes}


\subsection{Temperature Sensing Preformance}
\label{\detokenize{sensors/scd40:temperature-sensing-preformance}}

\begin{savenotes}\sphinxattablestart
\sphinxthistablewithglobalstyle
\centering
\begin{tabular}[t]{*{3}{\X{1}{3}}}
\sphinxtoprule
\sphinxstyletheadfamily 
\sphinxAtStartPar
Parameter
&\sphinxstyletheadfamily 
\sphinxAtStartPar
Conditions
&\sphinxstyletheadfamily 
\sphinxAtStartPar
Value
\\
\sphinxmidrule
\sphinxtableatstartofbodyhook
\sphinxAtStartPar
Range
&\begin{itemize}
\item {} 
\end{itemize}
&
\sphinxAtStartPar
\sphinxhyphen{}10°C \sphinxhyphen{} 60°C
\\
\sphinxhline
\sphinxAtStartPar
Accuracy
&
\sphinxAtStartPar
15°C \sphinxhyphen{} 35°C
&
\sphinxAtStartPar
\(\pm\)0.8°C
\\
\sphinxhline
\sphinxAtStartPar
Accuracy
&
\sphinxAtStartPar
\sphinxhyphen{}10°C \sphinxhyphen{} 60°C
&
\sphinxAtStartPar
\(\pm\)1.5°C
\\
\sphinxhline
\sphinxAtStartPar
Repeatability
&
\sphinxAtStartPar
Typical
&
\sphinxAtStartPar
\(\pm\)0.1°C
\\
\sphinxbottomrule
\end{tabular}
\sphinxtableafterendhook\par
\sphinxattableend\end{savenotes}


\subsection{Measuring CO2}
\label{\detokenize{sensors/scd40:measuring-co2}}
\sphinxAtStartPar
Measuring CO2 is another important parameter for indoor and outdoor air quality. Health Canada reccomends not exceeding 1000ppm of CO2 in a 24 hour period. A high concentration of CO2 can increase the risk of respiratory symptoms, decreased cognitive function, and neurophysiological symptoms such as headaches, tiredness, and dizzyness. The graph below shows different concentrations and what effects one may encounter at that concentration.

\sphinxhref{scd40.html}{\sphinxincludegraphics{{co2Limits}.jpg}}

\sphinxstepscope


\section{VEML7700}
\label{\detokenize{sensors/veml7700:veml7700}}\label{\detokenize{sensors/veml7700:id1}}\label{\detokenize{sensors/veml7700::doc}}
\sphinxhref{veml7700.html}{\sphinxincludegraphics{{MFG_VEML7700}.png}}

\sphinxAtStartPar
The VEML7700 high accuracy 16\sphinxhyphen{}bit light sensor can be used to precisely measure ambient light. Below is a functional block diagram of the device, where the sensor is an ALS photodiode.

\sphinxhref{veml7700.html}{\sphinxincludegraphics{{veml7700_blockDiagram}.png}}


\subsection{Specifications}
\label{\detokenize{sensors/veml7700:specifications}}

\begin{savenotes}\sphinxattablestart
\sphinxthistablewithglobalstyle
\centering
\begin{tabulary}{\linewidth}[t]{TTT}
\sphinxtoprule
\sphinxstyletheadfamily 
\sphinxAtStartPar
Parameter
&\sphinxstyletheadfamily 
\sphinxAtStartPar
Typical Value
&\sphinxstyletheadfamily 
\sphinxAtStartPar
Units
\\
\sphinxmidrule
\sphinxtableatstartofbodyhook
\sphinxAtStartPar
Digital resolution
&
\sphinxAtStartPar
0.0036
&
\sphinxAtStartPar
lux/bit
\\
\sphinxhline
\sphinxAtStartPar
Detectable minimum illuminance
&
\sphinxAtStartPar
0.0072
&
\sphinxAtStartPar
lux
\\
\sphinxhline
\sphinxAtStartPar
Detectable maximum illuminance
&
\sphinxAtStartPar
120,000
&
\sphinxAtStartPar
lux
\\
\sphinxhline
\sphinxAtStartPar
Dark offset
&
\sphinxAtStartPar
3
&
\sphinxAtStartPar
steps
\\
\sphinxbottomrule
\end{tabulary}
\sphinxtableafterendhook\par
\sphinxattableend\end{savenotes}

\sphinxAtStartPar
Below is a graph showing the spectral response of the sensor. It is sensitive to wavelengths ranging from 450nm \sphinxhyphen{} 650nm, normalized at 550nm.

\sphinxhref{veml7700.html}{\sphinxincludegraphics{{veml7700_spectralResponse}.png}}

\sphinxAtStartPar
Additionally, below is a graph showing the sensitivity of the white light channel on the sensor.

\sphinxhref{veml7700.html}{\sphinxincludegraphics{{veml7700_whiteSensitivitySpectrum}.png}}


\chapter{Hardware}
\label{\detokenize{index:hardware}}
\sphinxstepscope


\section{Destination Weather Station \sphinxhyphen{} Hardware}
\label{\detokenize{hardware/index:destination-weather-station-hardware}}\label{\detokenize{hardware/index:hardware}}\label{\detokenize{hardware/index::doc}}
\sphinxAtStartPar
The Destination Weather Station was designed in \sphinxhref{https://www.kicad.org/}{KiCad} and assembled by hand. the weather station includes several features that improve user experiance. These are:

\sphinxstepscope


\subsection{Power}
\label{\detokenize{hardware/power:power}}\label{\detokenize{hardware/power:id1}}\label{\detokenize{hardware/power::doc}}
\sphinxAtStartPar
The power subsystem of the weather station was designed with reverse voltage protection in mind. Typically, if a battery is connected to a circuit backwards, it will not work, or worse, fry the entire system. This is why reverse current protection is often added to circuits to prevent such user error. This allows the user to not have to worry about the polarity of their LiPo battery. On the Destination Weather Station v4.5 we have taken this one step further, by not only having reverse voltage/polarity protection, but also correcting this polarity. This works using the same principal as an \sphinxhref{https://en.wikipedia.org/wiki/Rectifier}{AC rectifier} circuit. This DC rectifier consists of two \sphinxhref{https://en.wikipedia.org/wiki/NMOS\_logic}{N\sphinxhyphen{}Channel MOSFETs} and two \sphinxhref{https://en.wikipedia.org/wiki/PMOS\_logic}{P\sphinxhyphen{}Channel MOSFETs}. These MOSFETs combine to essentially create a rectifier circuit, which will only allow the correct polarity to pass through the FETs. A schematic of this circuit can be found below.

\sphinxhref{power.html}{\sphinxincludegraphics{{dcRectifier}.png}}

\sphinxAtStartPar
\sphinxstylestrong{Note:} There is no over\sphinxhyphen{}current protection, so make sure to not exceed 5V in.

\sphinxstepscope


\subsection{Storage}
\label{\detokenize{hardware/storage:storage}}\label{\detokenize{hardware/storage:id1}}\label{\detokenize{hardware/storage::doc}}
\sphinxAtStartPar
The storage on the Destination Weather station is simply a microSD card slot to store recorded sensor data for further evalulation. There is also chip\sphinxhyphen{}detect functionality which allows the code to know if a card has been inserted or not.

\sphinxstepscope


\subsection{Human Interface Devices}
\label{\detokenize{hardware/hid:human-interface-devices}}\label{\detokenize{hardware/hid:hid}}\label{\detokenize{hardware/hid::doc}}
\sphinxAtStartPar
On the Destination Weather Station there are 4 buttons, which are the primary way to interact with the device. These buttons are each labeled with arrows ↑, ↓, ←, \(\rightarrow\). These buttons are used to scroll through the menus and to switch between the different data displays.

\sphinxstepscope


\section{Troubleshooting}
\label{\detokenize{faq/troubleshooting:troubleshooting}}\label{\detokenize{faq/troubleshooting:id1}}\label{\detokenize{faq/troubleshooting::doc}}
\sphinxAtStartPar
Coming soon!


\chapter{Software}
\label{\detokenize{index:software}}
\sphinxstepscope


\section{Software}
\label{\detokenize{software/index:software}}\label{\detokenize{software/index:id1}}\label{\detokenize{software/index::doc}}
\sphinxAtStartPar
Software for the Destination Weather Station v4.5.

\sphinxstepscope


\subsection{Blink Sketch}
\label{\detokenize{software/blink:blink-sketch}}\label{\detokenize{software/blink:id1}}\label{\detokenize{software/blink::doc}}
\sphinxAtStartPar
The \sphinxhref{https://gitlab.com/Destination-SPACE/ds-weather-station-v4.5/-/blob/main/software/Blink/Blink.ino}{Blink} sketch is used as a beginners program and to test the on\sphinxhyphen{}board RGB LED and NeoPixel.

\sphinxstepscope


\subsection{Sensor Test Sketch}
\label{\detokenize{software/sensor-test:sensor-test-sketch}}\label{\detokenize{software/sensor-test:sensor-test}}\label{\detokenize{software/sensor-test::doc}}
\sphinxAtStartPar
The \sphinxhref{https://gitlab.com/Destination-SPACE/ds-weather-station-v4.5/-/blob/main/software/Sensor\_Test/Sensor\_Test.ino}{Sensor Test} sketch is used to make sure the sensors on the weather station are working correctly. Use this sketch as the first program to upload after the \sphinxhref{https://gitlab.com/Destination-SPACE/ds-weather-station-v4.5/-/blob/main/software/Blink/Blink.ino}{Blink} sketch.

\sphinxstepscope


\subsection{Demo Sketch}
\label{\detokenize{software/demo:demo-sketch}}\label{\detokenize{software/demo:id1}}\label{\detokenize{software/demo::doc}}
\sphinxAtStartPar
The \sphinxhref{https://gitlab.com/Destination-SPACE/ds-weather-station-v4.5/-/blob/main/software/Demo/Demo.ino}{Demo} sketch is used as an introductory program to most of the features on the weather station. This includes scrollable menus for all sensors.

\sphinxstepscope


\subsection{Full Code Sketch}
\label{\detokenize{software/full-code:full-code-sketch}}\label{\detokenize{software/full-code:blink}}\label{\detokenize{software/full-code::doc}}
\sphinxAtStartPar
The \sphinxhref{https://gitlab.com/Destination-SPACE/ds-weather-station-v4.5/-/blob/main/software/Full\_Code/Full\_Code.ino}{Full Code} sketch incorporates all of the components of the weather station, including data recording. This has all the menus of the \sphinxhref{https://gitlab.com/Destination-SPACE/ds-weather-station-v4.5/-/blob/main/software/Demo/Demo.ino}{Demo} sketch as well as a menu used to start and stop data recording.


\chapter{Miscellaneous}
\label{\detokenize{index:miscellaneous}}
\sphinxstepscope


\section{Frequently Asked Questions}
\label{\detokenize{faq/index:frequently-asked-questions}}\label{\detokenize{faq/index:faq}}\label{\detokenize{faq/index::doc}}
\sphinxAtStartPar
Coming soon!



\renewcommand{\indexname}{Index}
\printindex
\end{document}